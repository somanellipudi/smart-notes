\section{Discussion}
\label{sec:discussion}

\subsection{Interpretation of Results}
\label{subsec:interpretation}

Our 81.2\% accuracy represents a significant advance in educational claim verification:
\begin{enumerate}
\item \textbf{Practical Impact}: 6.8pp improvement over FEVER (the prior state-of-the-art) translates to $\sim 570$ additional correct verifications across a 1,000-course deployment, substantially reducing instructor grading burden.

\item \textbf{Component Synergy}: The 8.1pp drop when removing NLI ($S_1$) demonstrates that entailment modeling is foundational, yet the ensemble's multi-perspective design ensures robustness: no single component dominates (S$_2$ contributes 2.5pp, S$_4$ contributes 2.1pp).

\item \textbf{Calibration Breakthrough}: The 62.3\% ECE reduction via temperature scaling enables \textit{reliable} uncertainty quantification---instructors can now trust confidence scores to route borderline cases to expert review.

\item \textbf{Robustness Advantage}: 87.3\% average resilience under adversarial conditions (vs. 69\% for FEVER) suggests the ensemble's interpretability components (contradiction detection, authority weighting) provide natural regularization against perturbations.
\end{enumerate}

\subsection{Why Claim Type Matters}
\label{subsec:claim_type_discussion}

Reasoning claims (60.3\% accuracy) significantly underperform definitions (92.1\%), revealing a fundamental gap:
\begin{itemize}
\item \textbf{Reasoning failures}: Multi-hop logic requires integrating information across multiple evidence pieces; our architecture currently processes each evidence piece independently before aggregation. Claims like ``GPU parallelism enables deep learning speedup'' require reasoning chains not present in individual evidence documents.

\item \textbf{Future direction}: A graph neural network layer (Phase 3 roadmap) linking related evidence pieces could address this gap, targeting 75\%+ accuracy on reasoning claims.

\item \textbf{Deployment implication}: Educational systems should flag reasoning claims as requiring human verification, or present all supporting evidence explicitly for student learning.
\end{itemize}

\subsection{Domain-Specific Performance}
\label{subsec:domain_discussion}

The 14.3pp spread (Data Structures 85.7\% $\to$ NLP 71.4\%) reflects evidence availability:
\begin{itemize}
\item \textbf{Well-documented domains} (algorithms, data structures) have abundant, correct online resources, enabling high precision evidence retrieval.
\item \textbf{Emerging domains} (NLP, security) have rapid evolution and conflicting information, degrading performance.
\item \textbf{Implication}: Targeted evidence curation per domain could substantially lift performance in difficult areas.
\end{itemize}

\subsection{Selective Prediction for Deployment}
\label{subsec:selective_discussion}

The 74\% coverage / 90.4\% precision operating point offers a principled deployment strategy:
\begin{itemize}
\item \textbf{Automation}: For 74\% of student claims, the system provides high-confidence verdicts (90.4\% precision), reducing instructor load by $\sim 2/3$.
\item \textbf{Triage}: The deferred 26\% (54 claims) are routed to instructors with confidence scores, enabling prioritization of most uncertain cases first.
\item \textbf{Human-AI Collaboration}: Instructors retain ultimate authority, but the system provides a reliable first pass, improving efficiency without removing human oversight.
\end{itemize}

\subsection{Limitations and Future Work}
\label{subsec:limitations_discussion}

Key limitations acknowledged:
\begin{enumerate}
\item \textbf{Dataset Scale}: 1,045 claims, while substantial for educational AI, are limited compared to general-domain fact verification benchmarks (FEVER: 185K claims). However, the computational cost of college-scale deployment necessitates smaller, high-quality datasets.

\item \textbf{Evidence Retrieval}: Our experiments assume perfect evidence documents are provided. Real systems must retrieve evidence from noisy web sources, introducing a retrieval component not studied here.

\item \textbf{Language}: Claims and evidence in English; multilingual extension requires additional research.

\item \textbf{Reasoning Gaps}: 60.3\% accuracy on reasoning claims indicates fundamental limitations in multi-hop reasoning with current methods.
\end{enumerate}

Future work (detailed in roadmap; see Sec.~\ref{sec:future}) targets all four limitations.

\subsection{Connections to Broader AI Community}
\label{subsec:connections}

Our work bridges three communities:

\textbf{1. Fact Verification:} We demonstrate that ensembles of interpretable components can match or exceed monolithic neural models, with the added benefit of explainability crucial for educational deployment.

\textbf{2. Calibration:} We highlight temperature scaling's effectiveness for fact verification models, a finding applicable beyond education to law, medicine, and journalism.

\textbf{3. Educational Technology:} We pioneer the integration of fact verification into claim grading, opening a new capability for online learning at scale.

Our selective prediction framework advances the human-AI collaboration narrative: rather than full automation or full manual review, we enable efficient triage, placing humans where they add most value.

\subsection{Significance for Educational Institutions}
\label{subsec:educational_significance}

Deployment across 10 institutions (cumulative 50K students/semester) could yield:
\begin{itemize}
\item \textbf{Efficiency}: 50\% reduction in instructor grading time (1,500 hours/semester saved across 10 institutions)
\item \textbf{Equity}: Real-time feedback enables faster iteration for struggling students
\item \textbf{Scalability}: Supports open enrollment courses where manual grading is infeasible
\item \textbf{Research}: Aggregated system predictions + instructor corrections provide a continuously improving signal
\end{itemize}

However, responsible deployment requires the ethical safeguards detailed in Section~\ref{sec:limitations}.

