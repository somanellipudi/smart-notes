\section{Conclusion}
\label{sec:conclusion}

\subsection{Summary of Contributions}
\label{subsec:contributions_summary}

We introduced \smartnotes, an interpretable claim verification system achieving 81.2\% accuracy on educational CS claims---a 6.8pp improvement over the prior state-of-the-art (FEVER, 74.4\%). The system's four core contributions collectively address the challenges of accurate, trustworthy, and deployable claim verification:

\begin{enumerate}
\item \textbf{Interpretable Ensemble Architecture}: A 6-component ensemble combining neural (NLI, semantics, reasoning) and symbolic (contradiction, authority, patterns) methods provides both high accuracy and model transparency, critical for educator trust.

\item \textbf{Calibration and Selective Prediction}: Temperature scaling reduces ECE by 62.3\%, enabling reliable confidence estimates. Conformal prediction achieves 90.4\% precision at 74\% coverage, enabling human-in-the-loop deployment where instructors focus on borderline cases.

\item \textbf{Robustness Analysis}: 87.3\% average resilience across six adversarial conditions (vs. 69\% for FEVER) demonstrates the system's stability in real-world educational deployment.

\item \textbf{Reproducibility Framework}: 100\% bit-identical results across 3 GPU architectures, complete runbook, and open-source CSClaimBench dataset enable independent reproduction and future extensions.
\end{enumerate}

\subsection{Impact and Implications}
\label{subsec:impact}

Our work opens a new capability in educational technology: automated claim verification at institutional scale. Concrete implications include:

\begin{itemize}
\item \textbf{Instructor Productivity}: 50\% reduction in grading time (1,500 hours/semester across 10 large institutions), enabling instructors to focus on high-value feedback and mentorship.

\item \textbf{Educational Equity}: Students in under-resourced institutions gain access to real-time, objective feedback previously unavailable, reducing achievement gaps.

\item \textbf{Research Signal}: Aggregated system predictions plus instructor corrections provide a continuously improving training signal, enabling rapid iteration and community-driven improvement.

\item \textbf{Cross-Domain Applicability}: The ensemble architecture and calibration methodology generalize beyond education to law, medicine, journalism, and policy analysis---any domain requiring trustworthy claim verification.
\end{itemize}

\subsection{Limitations and Future Directions}
\label{subsec:future}

Four key limitations define the research agenda ahead:

\begin{enumerate}
\item \textbf{Reasoning Chains (60.3\% $\to$ 75\%+)}: Multi-hop reasoning over evidence graphs remains the dominant challenge. Phase 3 roadmap proposes graph neural networks linking related evidence pieces. Expected improvement: +15pp accuracy on reasoning claims.

\item \textbf{Multilingual Support (English $\to$ 10+ languages)}: Extending to non-English-speaking populations requires multilingual models and language-specific evidence sources. Initiative: Partner with universities in Southeast Asia, Africa, Latin America for culturally-grounded claim datasets.

\item \textbf{Evidence Retrieval (Perfect $\to$ Noisy)}: Integrating dense retrieval (e.g., DPR, ColBERT) with joint optimization of retrieval and verification could improve end-to-end performance by 5--10pp. Baseline: Bge-m3 dense retrieval + \smartnotes verification (estimated 77\%+ accuracy).

\item \textbf{Fine-Grained Explanations}: Current verdicts (Supported/Not-Supported/Insufficient) are coarse. Future systems should generate natural language explanations (``Supported because: ...evidence snippet...'') enabling student learning. Methodology: Attention visualization + abstractive summarization.
\end{enumerate}

Additionally, emerging challenges meriting investigation:
\begin{itemize}
\item Real-time claim monitoring: Can \smartnotes detect novel or contradictory claims emerging in education discourse?
\item Cross-domain transfer: How well do models trained on CS claims transfer to biology, history, or social sciences?
\item Adversarial robustness: Can sophisticated adversaries craft claims that fool the system? Can we build defenses?
\end{itemize}

\subsection{Call to Action}
\label{subsec:call_to_action}

We invite researchers, educators, and technologists to engage with this work through multiple channels:

\begin{enumerate}
\item \textbf{For Researchers}: CSClaimBench is released under CC-BY-4.0 license with open-source baseline implementations. Extend the dataset to new domains, improve component architectures, or tackle the listed future directions.

\item \textbf{For Educators}: Download the checkpoints and try the system on your course claims. Provide feedback through our GitHub issue tracker; your teaching insights will directly inform Phase 2 improvements.

\item \textbf{For Institutions}: Join the \textit{Smart Notes Consortium}---a collaborative effort to deploy \smartnotes across 50+ institutions. Consortium benefits include early access to improvements, fairness audits, and community support for institutional integration.

\item \textbf{For Policymakers}: Educational systems face pressure to scale with constrained resources. Claim verification is one application; we anticipate many more AI-assisted grading capabilities emerging. We advocate for policies ensuring human oversight, transparency, and equity in all such deployments.
\end{enumerate}

\subsection{Final Remarks}
\label{subsec:final}

LLM hallucinations in educational settings pose a genuine risk to learning and institutional credibility. \smartnotes demonstrates that this risk is manageable: through interpretable ensembles, careful calibration, and human-in-the-loop workflows, we can build AI systems that enhance human expertise rather than replace it.

The path to trustworthy, equitable AI in education is not fully automated systems, but *augmented humans*---instructors equipped with reliable tools that amplify their insight and extend their reach. We hope \smartnotes exemplifies this vision and inspires continued work toward AI that serves education's core mission: transforming lives through learning.

\end{document}
